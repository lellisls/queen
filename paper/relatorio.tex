\documentclass[conference]{IEEEtran}
\usepackage[brazilian]{babel}
\usepackage[utf8]{inputenc}
\usepackage[T1]{fontenc}
\usepackage{cite}
\ifCLASSINFOpdf
  \usepackage[pdftex]{graphicx}
  % declare the path(s) where your graphic files are
  % \graphicspath{{../pdf/}{../jpeg/}}
  % and their extensions so you won't have to specify these with
  % every instance of \includegraphics
  % \DeclareGraphicsExtensions{.pdf,.jpeg,.png}
\else
  % or other class option (dvipsone, dvipdf, if not using dvips). graphicx
  % will default to the driver specified in the system graphics.cfg if no
  % driver is specified.
  \usepackage[dvips]{graphicx}
  % declare the path(s) where your graphic files are
  % \graphicspath{{../eps/}}
  % and their extensions so you won't have to specify these with
  % every instance of \includegraphics
  % \DeclareGraphicsExtensions{.eps}
\fi
\usepackage[cmex10]{amsmath}
\usepackage{algorithmic}
\usepackage{algorithm}
\usepackage{array}
\usepackage{mdwmath}
\usepackage{mdwtab}
%\usepackage{eqparbox}
\usepackage[tight,footnotesize]{subfigure}
%\usepackage[caption=false]{caption}
\usepackage[font=footnotesize]{subfig}
%\usepackage{stfloats}
\usepackage{url}

\hyphenation{op-tical net-works semi-conduc-tor}

\begin{document}
\title{Classificação de alimentos segmentados}

\author{
\IEEEauthorblockN{Lucas Santana Lellis - R.A.: 69618}
\IEEEauthorblockA{Instituto de Ciência e Tecnologia\\
	Universidade Federal de São Paulo} }

% make the title area
\maketitle

%\IEEEpeerreviewmaketitle

\section{Introdução}
\IEEEPARstart
O problema das 8 rainhas consiste em posicionar 8 rainhas em um tabuleiro 
8x8 de forma que elas não se cruzem em nenhuma direção.

Trata-se de um problema bastante complexo para uma busca em profundidade, 
já que são possíveis 4.426.165.368 estados diferentes, porém apenas 92 soluções. \cite{wiki8queen}

Existem diversas formas de se resolver este problema, incluindo soluções de inteligência artificial,
 como a \textbf{subida de encosta} e os \textbf{algoritmos genéticos}.
\subsection{Subida de encosta}
Segundo \cite{norvig2014inteligencia}, os algoritmos de subida de encosta baseiam-se em um loop simples
a fim de maximizar( ou minimizar ) um valor. Trata-se de uma busca gulosa, que tem como condição de parada 
a descoberta de um valor máximo( ou mínimo ) local ( ou global ), eles param de iterar.

O valor, representado pela "colina", pode ser a \textbf{função de utilidade} que define a solução do
problema em questão. A cada passo são avaliados os diferentes estados possíveis, sendo escolhida sempre a de maior utilidade.

Este algoritmo apresenta diferentes problemas, como:


\begin{itemize}
	\item O \textbf{máximo local}, (ou mínimo local), em que o valor encontrado não é
	a solução ideal do problema. 
	\item As \textbf{cordilheiras} são sequencias de máximos locais, o que torna difícil a navegação 
	para algoritmos gulosos.
	\item Os \textbf{platôs} são áreas planas da topologia de espaço de estados. Nestes trechos o algoritmo fica parado, uma vez que não existe um caminho superior para que o algoritmo continue a subir.
\end{itemize} 

Por estes motivos, este algoritmo possui uma alta taxa de erros, embora seja eficiente quando acerta.

 No caso do problema das 8 rainhas, pretende-se \textbf{minimizar} o número total de colisões entre as rainhas.


\subsection{Algoritmo Genético}
Segundo \cite{norvig2014inteligencia}, um algoritmo genético é uma variante do algoritmo em feixe estocástica
na qual os sucessores são gerados a partir da combinação de dois estados pais.

Este algoritmo baseia-se em conceitos de seleção natural, em que os seres evoluem a partir do cruzamento e mutação
dos genes. Estes problemas baseiam-se em conceitos como "variabilidade genética", e de que "os melhores sobrevivem".

O algoritmo possui três etapas principais:
\begin{itemize}
	\item \textbf{Geração de problemas: } É o passo em que a população inicial é gerada randomicamente. 
	\item \textbf{Cruzamento: } O cruzamento entre cromossomos, em que os vetores que os representam são misturados em um \textit{k}.
	\item \textbf{Mutação: } A mutação altera a informação em alguma parte do vetor, promovendo variações que podem ajudar a alcançar resultados inacessíveis a partir dos estados atuais.
\end{itemize}

Este algoritmo também procura maximizar uma função de utilidade, e pode ser implementado com diferentes abordagens.
\hfill São José dos Campos, 24 de Abril de 2015

%
%\begin{figure}[!t]
%\centering
%\includegraphics[width=2.5in]{myfigure}
% where an .eps filename suffix will be assumed under latex, 
% and a .pdf suffix will be assumed for pdflatex; or what has been declared
% via \DeclareGraphicsExtensions.
%\caption{Simulation Results}
%\label{fig_sim}
%\end{figure}

%\begin{figure*}[!t]
%\centerline{\subfloat[Case I]\includegraphics[width=2.5in]{subfigcase1}%
%\label{fig_first_case}}
%\hfil
%\subfloat[Case II]{\includegraphics[width=2.5in]{subfigcase2}%
%\label{fig_second_case}}}
%\caption{Simulation results}
%\label{fig_sim}
%\end{figure*}

%\begin{table}[!t]
%% increase table row spacing, adjust to taste
%\renewcommand{\arraystretch}{1.3}
% if using array.sty, it might be a good idea to tweak the value of
% \extrarowheight as needed to properly center the text within the cells
%\caption{An Example of a Table}
%\label{table_example}
%\centering
%% Some packages, such as MDW tools, offer better commands for making tables
%% than the plain LaTeX2e tabular which is used here.
%\begin{tabular}{|c||c|}
%\hline
%One & Two\\
%\hline
%Three & Four\\
%\hline
%\end{tabular}
%\end{table}

\section{Objetivos}
Este projeto tem como objetivo implementar e comparar os algoritmos de \textbf{subida de encosta} e \textbf{genético}.
Comparando também diferentes configurações do algoritmo genético.
\section{Metodologia Experimental}
O projeto foi implementado inteiramente em \textit{Python}, de forma que ambos os algoritmos compartilhassem dos mesmos
geradores de problemas (\textbf{generate}), e das mesmas funções de utilidade (\textbf{collisions}). Tais funções estão presentes no arquivo "\textit{lucaslellis\_69618\_common.py}".
\subsection{Algoritmo de subida de encosta}
O algoritmo de subida de encosta foi feito segundo a especificação de Russel\cite{norvig2014inteligencia}, e se dá, basicamente de acordo com o algoritmo 1. O problema foi divido em diferentes funções para fragmentar o problema, sendo estas:
\begin{itemize}
	\item \textbf{Sucessors:} Função que recebe estado atual e retorna um vetor com todos os estados possíveis a partir
	do estado atual. Os estados são gerados somando ou subtraindo a posição de cada uma das rainhas.
	\item \textbf{Best:} Função que recebe o vetor de estados sucessores. Ela calcula e compara o número de colisões
	de cada um dos estados, e retorna o estado com o menor número de colisões.
\end{itemize}
\begin{algorithm}
\caption{Algoritmo de Subida de Encosta}
\begin{algorithmic}[1]
	\STATE n = 8 
	\STATE vec = generate(n)
	\WHILE {collisions(vec) $\neq$ 0}
	\STATE s = sucessors(vec)
	\STATE b = best(s)
	\IF {( collisions(b) $\leq$ collisions(vec) )}
	\STATE break
	\ELSE
		\STATE	vec = b
	\ENDIF
	\ENDWHILE
\end{algorithmic} 
\end{algorithm}
O código referente ao algoritmo está presente no arquivo "\textit{lucaslellis\_69618\_hillclimb.py}". 
\subsection{Algoritmo genético}
O algoritmo genético é um problema complexo e foi modelado utilizando classes. Após diversas tentativas em torná-lo
robusto, a implementação final diferencia-se da definida por Russel\cite{norvig2014inteligencia} em alguns aspectos.\\
Primeiramente, a etapa de cruzamento gera dois descendentes, ao invés de um. Além disso, o algoritmo mantém uma porcentagem dos melhores genes da geração anterior, garantindo a evolução dos mais fortes, e reduzindo a chance de uma piora nos resultados. Por último, os cruzamentos não são limitados por uma taxa modificável, uma vez que são realizados para complementar o espaço deixado pelos genes "mortos" da geração anterior.\\
O algoritmo se dá, basicamente de acordo com o algoritmo 2. O problema, ainda mais complexo, foi fragmentado em diversas funções, listadas a seguir:
\begin{itemize}
	\item \textbf{crossOver:} Função responsável pela mistura de dois genes (vetores), dado um ponto k.
	\item \textbf{mutate:} Função utiliza um número aleatório entre 0 e 7 para definir uma posição do vetor, e substitui o valor nesta posição por um outro valor aleatório diferente do atual.
	\item \textbf{calcFitness:} Calcula o valor de 28 - crossover, uma vez que para o algoritmo genético procuramos maximixar uma função de utilidade.
	\item \textbf{fillPopulation:} Função responsável por gerar um vetor de genes arbitrários.
	\item \textbf{sortPopulation:} Função responsável por ordenar um vetor de genes em ordem decrescente de utilidade.
	\item \textbf{updateFitness:} Atualiza a utilidade de toda a população.	
	\item \textbf{select:} Seleciona dois genes diferentes de uma população.
\end{itemize}

Além disso, existem algumas variáveis configuráveis:

\begin{itemize}
%	populationSize, keepRate,  crossRate, mutationRate, maxIterations
	\item \textbf{populationSize:} Tamanho fixo da população.
	\item \textbf{keepRate:} Número de "sobreviventes" à cada geração.
	\item \textbf{mutationRate:} Probabilidade da mutação.
	\item \textbf{maxIterations:} Número máximo de gerações.
\end{itemize}

\begin{algorithm}
	\caption{Algoritmo Genético}
	\begin{algorithmic}[1]
		\REQUIRE populationSize, keepRate, mutationRate, maxIterations
		\STATE population = []
		\STATE fillPopulation(population,populationSize)
		\STATE updateFitness(population)
		\STATE sortPopulation( population )
		\STATE newPopulationSz = int(populationSize * keepRate) + int(populationSize * keepRate)\%2
		\STATE finished = \FALSE
		\STATE boardSize = 8
		\STATE crossPoint = rand(0,boardSize)
		\STATE count = 0
		\STATE maxFitness = 28
		\WHILE {finished == \FALSE}
		\STATE count = count + 1
		\STATE crossPoint = (crossPoint+1)\%boardSize
		population = population[0 : newPopulationSz ]
		\WHILE {length(population) < populationSize }
		
		%\IF {( collisions(b) $\leq$ collisions(vec) )}
		%\STATE a = 1
		%\ENDIF
		\STATE gene1, gene2 = select(population)
		\STATE gene1, gene2 = crossOver(gene1,gene2,crossPoint)
		\STATE population.append(gene1)
		\STATE population.append(gene2)
		\ENDWHILE
		\FOR {gene in population} 
		\IF{random(0,100) < mutationRate }
		\STATE mutate(gene)
		\ENDIF
		\ENDFOR
		\STATE updateFitness(population)
		\IF{fitness(best) == maxFitness \OR count $\geq$ maxIterations }
		\STATE finished = \TRUE 
		\ENDIF
		\ENDWHILE
	\end{algorithmic} 
\end{algorithm}
O código referente ao algoritmo está presente no arquivo "\textit{lucaslellis\_69618\_genetic.py}".

\subsection{Análise estatística}
Após implementados e testados, os algoritmos foram executados exaustivamente para a obtenção de dados para posterior
comparação. Foram anotados o número de passos ( ou gerações ) necessários para finalizar o processamento, e o número de
colisões encontrados ao final. O número de colisões ser maior do que zero significa que a simulação falhou em chegar a um
resultado.\\
Para encontrar a melhor configuração do algoritmo genético, o programa foi executado 500 vezes com 180 diferentes
combinações de parâmetros, totalizando 90000 execuções. Para limitar o tempo de execução, o algoritmo foi limitado a 1000 gerações.

\section{Resultados e Discussão}


\begin{thebibliography}{2}
\bibitem{wiki8queen}
Wikipedia, Eight queens puzzle - Disponível em: 
 http://en.wikipedia.org/wiki/Eight\_queens\_puzzle
\bibitem{norvig2014inteligencia}
Stuart Russell, Peter Norvig, and Artificial Intelligence. A modern
approach. \textit{Artificial Intelligence. Prentice-Hall, Egnlewood Cliffs}, 25,
1995.
\end{thebibliography}

\end{document}


